\documentclass[line]{res}

\usepackage{geometry}
\usepackage[dvipsnames]{xcolor}
\usepackage[utf8]{inputenc}
\usepackage{fontenc}
\usepackage{enumitem}
\usepackage[colorinlistoftodos]{todonotes}
\usepackage{hyperref}
\usepackage{multicol}

\hypersetup{
  colorlinks=true,
  linkcolor=CadetBlue,
  filecolor=CadetBlue,
  urlcolor=CadetBlue,
}

\geometry{
  a4paper,
  total={210mm,297mm},
  left=15mm,
  right=15mm,
  top=13mm,
  bottom=13mm,
  bindingoffset=0mm
}
\setlist{nolistsep,left=1em}
\newsectionwidth{0in}

\begin{document}

\name{{\huge Abraham Gonzalez}}
\address{\href{https://abejgonzalez.github.io}{https://abejgonzalez.github.io} $|$ \href{mailto:abe.gonzalez@berkeley.edu}{abe.gonzalez@berkeley.edu}}

\begin{resume}

\vspace{-8mm}

\section{\Large{Education}}
\phantomsection
\label{sec:education}
\vspace{1mm}

\textbf{University of California, Berkeley} \hfill Expected July 2025
\\
\textit{Ph.D. in Electrical Engineering and Computer Sciences}, advised by Krste Asanovi\'c, GPA: 3.96/4.0
\\
Dissertation Title: ``End-to-end Heterogeneous System Design for Hyperscale Big Data Processing''

\vspace{-3mm}

\textbf{The University of Texas at Austin} \hfill August 2014 - May 2018
\\
\textit{B.S. in Electrical and Computer Engineering}, GPA: 3.98/4.0

\vspace{-3mm}

\section{\Large{Work Experience}}
\phantomsection
\label{sec:jobs}
\vspace{1mm}

\textbf{Graduate Student Researcher} \hfill August 2018 - Present
\\
University of California, Berkeley | Berkeley, CA
\\
\vspace{-3mm}
\begin{itemize}
\item Advised by Professor Emeritus and Professor of the Graduate School Krste Asanovi\'c.
\item Ph.D. candidate and \href{https://adept.eecs.berkeley.edu/}{ADEPT}/\href{https://slice.eecs.berkeley.edu/}{SLICE lab} member researching hyperscale cloud architectures, accelerator scheduling, and hardware design methodologies.
\item Co-lead of the Hyperscale system-on-chip (SoC) project focused on hyperscale hardware/software co-design.
\item Co-lead of the \href{https://github.com/ucb-bar/chipyard}{Chipyard SoC framework} \href{https://scholar.google.com/scholar?cites=4549882523608568335&as_sdt=2005&sciodt=0,5&hl=en}{(used in 20+ tape-outs and 150+ pubs. @ 65+ companies/universities)}.
\item Co-lead of the award-winning \href{https://fires.im/}{FireSim FPGA-accelerated simulator} \href{https://fires.im/publications/#userpapers}{(used in 60+ pubs. @ 25+ companies/universities)}.
\item Developer of the \href{https://boom-core.org/}{Berkeley Out-of-Order Machine (BOOM)}, the first open-source RISC-V out-of-order core.
\item Led the tape-out/bring-up of the first \hyperref[sec:chipyard]{Chipyard} test chip, a \href{https://ieeexplore.ieee.org/abstract/document/9567768}{106.1 GOPS/W heterogeneous SoC in Intel 22FFL}.
\item \href{https://scholar.google.com/citations?user=dsAQJ4cAAAAJ&hl=en}{Published research and tape-outs} at top conferences including ISCA, DAC, IEEE MICRO, and ESSCIRC.
\item Lead organizer for \href{https://fires.im/blog/}{10+ tutorials and workshops} with 200+ unique attendees at top conferences.
\end{itemize}

\vspace{-3mm}

\textbf{Student Researcher Intern} \hfill June 2021 - July 2024
\\
Google | Sunnyvale, CA
\\
\vspace{-3mm}
\begin{itemize}
\item Student researcher working with Parthasarathy Ranganathan (Vice President and Engineering Fellow).
%\item Researched data analytics and RPC acceleration under Partha Ranganathan and Jichuan Chang.
\item Collaborated with the \href{https://techsysinfra.google/research/}{SystemsResearch@Google (SRG)} and Systems Infrastructure Performance teams.
\item Researched data processing and remote procedure call (RPC) optimizations as part of the \hyperref[sec:hyperscale-soc]{Hyperscale SoC project}.
\item Published first of its kind research on \href{https://dl.acm.org/doi/10.1145/3579371.3589082}{hyperscale big data processing characterization at ISCA '23}.
\item Open-sourced \href{https://github.com/google/fleetbench/tree/cd20746b68b307b148a761c676d6400f2541082d/fleetbench/rpc}{HyperRPCBench}, a novel representative RPC benchmark suite, with the \href{https://github.com/google/fleetbench}{Fleetbench} team.
\end{itemize}

\vspace{-3mm}

\textbf{Silicon Engineering Group Intern} \hfill June 2020 - August 2020
\\
Apple | Cupertino, CA
\\
\vspace{-3mm}
\begin{itemize}
\item Engineering intern working with Si-En Chang (Distinguished Engineer).
\item Developed computer architecture tooling for CPU verification.
\end{itemize}

\vspace{-3mm}

\textbf{Scalable Performance CPU Development Group Intern} \hfill May 2018 - August 2018
\\
Intel | Austin, TX
\\
\vspace{-3mm}
\begin{itemize}
\item Worked on debugging tools for the microcontroller integration team with senior engineers.
\item Setup novel workflows and infrastructure between the firmware and microcontroller integration teams for agility.
\end{itemize}

\vspace{-3mm}

\textbf{Office Shared Graphics Explore Intern} \hfill May 2016 - August 2016
\\
Microsoft | Redmond, WA
\\
\vspace{-3mm}
\begin{itemize}
\item Developed the proof-of-concept ``Sketchy Lines'' feature (now publicly available) in the Office suite using C++.
\item Investigated and coordinated new feature sets with senior engineers, program managers, and customers.
\end{itemize}

\vspace{-3mm}

\textbf{UIM Driver Intern} \hfill May 2015 - August 2015
\\
Qualcomm | San Diego, CA
\\
\vspace{-3mm}
\begin{itemize}
\item Designed a software framework for smartcard (UIM) interaction in C++/CLI and C++ with senior engineers.
\item Integrated designed framework into a .NET application managing smartcards via CCID.
\end{itemize}

\vspace{-3mm}

\section{\Large{Skills}}
\phantomsection
\label{sec:skills}
\vspace{1mm}

\begin{multicols}{2}
\textbf{Programming Languages}
\\
Scala, C/C++, Python, Chisel, SQL, \{System\}Verilog, RISC-V, Bash, Make, Bazel, TCL, TensorFlow, PyTorch

\columnbreak

\textbf{Other}
\\
Git, MapReduce, AWS, Google Cloud, Xilinx Virtex/UltraScale+ FPGAs, Cadence EDA tooling

\end{multicols}

\vspace{-7mm}

% \section{Professional Leadership and Membership}
% Member of LAGSES (Fall 2018-Pres.)\\
% Vice President (Spr. 2018), Corres. Secretary (Fall 2017), and member (Spr. 2016-Pres.) of HKN Honor Society\\
% %Member of HKN Honor Society (Spr. 2016-Now)\\
% %Equal Opportunity in Engineering (EOE) Pi tutor (Fall 2015, Fall 2017)\\
% Academic Director (Fall 2016-Fall 2017), and member (Fall 2014-Pres.) of Society of Hispanic Professional Engineers\\
% %3DS Austin Organizer Committee member (Fall 2014-Fall 2015) and participant (Fall 2014)\\

\section{\Large{Honors, Awards, and Selections}}
\phantomsection
\label{sec:honors}
\vspace{2mm}

\textbf{UC Berkeley:} DARPA Riser (Fa'22), Analog Devices Outstanding Designer (Sp'20), Berkeley Fellowship (Fa'18), EECS Excellence Award (Fa'18), GEM Fellowship (Sp'18), NSF GRFP Honorable Mention (Sp'18) \\
\textbf{UT Austin and prior:} Highest Honors (Sp'18),
Distinguished College Scholar (Sp'17/18),
College Scholar (Sp'16),
Roberto Rocca Scholarship (Fa'17),
Victor L. Hand Scholarship (Fa'16),
TI Diversity Scholarship (Fa'15),
2\textsuperscript{nd} Best Poster SHPE National Conference (Fa'17),
Qualcomm DECA Attendee (1/51 selected nationally) (Sp'15),
EOE/SHPE Freshman Academic Excellence Winner (Sp'15)

\end{resume}

\end{document}
